\documentclass[a4paper, 12pt]{article}

\usepackage[utf8]{inputenc}
\usepackage[spanish]{babel}
\usepackage[T1]{fontenc}
\usepackage{tikz}
\usepackage{fancyhdr}
\usepackage{graphicx}
\usepackage{minted}
\usepackage[a4paper, total={6.8in, 8in}]{geometry}
\usetikzlibrary{positioning}

\pagestyle{fancy}
\fancyhf{}
\rhead{Martín Romera Sobrado}
\chead{EPED}
\lhead{Tema 3}
\cfoot{\thepage}

\setlength{\headheight}{52pt}

\begin{document}

    \title{\textbf{Ejercicios Propuestos Tema 3}}
    \author{Martín ``n3m1dotsys'' Romera Sobrado}
    \maketitle

    \section{Ejercicio 1}

        \textit{Calcular el coste del Algoritmo de Euclides}

        \begin{minted}{java}
    public int mcd (int A, int B) {
        if (B == 0) return A;
        else return mcd(B, A % B);
    }
        \end{minted}

        En el mejor de los casos \texttt{B} será $0$ y la respuesta será 
        inmediata, sin embargho si entra en el bucle de llamadas recursivas del 
        \texttt{else} es más dificil analizar cuantas vueltas va a dar el 
        algortimo ya que la operación módulo (\texttt{\%} en \textit{Java}) 
        puede dar un resultado entre $0$ y \texttt{B}$-1$, según el valor de 
        $A$ en ese momento. Podemos analizar el peor caso del algoritmo que 
        según el \textit{teorea de Lamé} es tener como entrada dos número 
        sucesivos de la \textit{secuencia de Fibonacci}. Prblemos como 
        haría el algoritmo para la entrada \texttt{mcd(55,34)}:
        
        \begin{table}[ht!]
            \centering
            \begin{tabular}{|c|c|c|}
                \hline
                10 & \texttt{mcd(55,34)} & = \texttt{mcd(34,21)} \\\hline
                9  & \texttt{mcd(34,21)} & = \texttt{mcd(21,13)} \\\hline
                8  & \texttt{mcd(21,13)} & = \texttt{mcd(13,8 )} \\\hline
                7  & \texttt{mcd(13,8 )} & = \texttt{mcd(8 ,5 )} \\\hline
                6  & \texttt{mcd(8 ,5 )} & = \texttt{mcd(5 ,3 )} \\\hline
                5  & \texttt{mcd(5 ,3 )} & = \texttt{mcd(3 ,2 )} \\\hline
                4  & \texttt{mcd(3 ,2 )} & = \texttt{mcd(2 ,1 )} \\\hline
                3  & \texttt{mcd(2 ,1 )} & = \texttt{mcd(1 ,1 )} \\\hline
                2  & \texttt{mcd(1 ,1 )} & = \texttt{mcd(1 ,0 )} \\\hline
                1  & \texttt{mcd(1 ,0 )} & = \textbf{1} \\\hline
            \end{tabular}
            \caption{Ejecución de \texttt{mcd(55,34)}}
        \end{table}

        Vemos que recorre todas las posibles entradas de números sucesivos de la 
        \textit{secuencia de Fibonacci}, y que siendo este el peor caso teniendo 
        en cuenta que incrementará una iteración cada vez que \texttt{A} (y 
        \texttt{B}) lleguen al siguiente número de la secuencia, deducimos que 
        el orden de ejecución será a la inversa de del orden de crecimiento de 
        la \textit{secuencia de Fibonacci}, que se encuentra en un orden 
        logarítmico $O(\log n)$.

    \section{Ejercicio 2}

        \textit{Calcular el coste del algoritmo ``multiplicación rusa''}

        \begin{minted}{java}
    public int multRusa(int A, int B:) {
        if (A == 1) return B;
        if (A % 2 != 0) return B + multRusa(A/2, B*2);
        else return multRusa(A/2, B*2);
    }
        \end{minted}

        Este método consiste en realizar la suma de los valores de \texttt{B} 
        multiplicados por las potencias de $2$ que componen a \texttt{A}. Por 
        ejemplo, si tenemos que \texttt{A} es $42$ y \texttt{B} es $50$. La 
        descomposición en base $2$ de $42$ sería
        \[
            42 = 2^1 + 2^3 + 2^5 = 2 + 8 + 32  
        \]
        , de forma que con el método de multiplicación rusa sería:
        \[
            42 * 50 = 2^1 \cdot 50 + 2^3 \cdot 50 + 2^5 \cdot 50  
                    = 2 \cdot 50 + 8 \cdot 50 + 32 \cdot 50 = 2100          
        \]
        Cada producto representa una iteración en la que que se cumplía alguno 
        de los dos \texttt{if}-s. De todas formas también se itera cuando no se 
        realizan sumas. Una representación más afín del algoritmo podría ser la 
        siguiente tabla: 
        
        \begin{table}[ht!]
            \centering
            \begin{tabular}{|c|c|c|c|}
                \hline
                $2^0$ & \texttt{multRusa(42,50)}  = &       & \texttt{multRusa(21,100)} \\\hline
                $2^1$ & \texttt{multRusa(21,100)} = & 100 + & \texttt{multRusa(10,200)} \\\hline
                $2^2$ & \texttt{multRusa(10,200)} = &       & \texttt{multRusa(5 ,400)} \\\hline
                $2^3$ & \texttt{multRusa(5 ,400)} = & 400 + & \texttt{multRusa(2 ,800)} \\\hline
                $2^4$ & \texttt{multRusa(2 ,800)} = &       & \texttt{multRusa(1,1600)} \\\hline
                $2^5$ & \texttt{multRusa(1,1600)} = & 1600  & \textbf{end}\\\hline 
            \end{tabular}
            \caption{Ejecución de \texttt{multRusa(42,50)}}
        \end{table}

        Tiene que hacer tantas iteraciones como bits se necesitan para codificar 
        \texttt{A} en base 2, es decir:
        \[
            T(n) = \log_2(n) \in O(\log n)  
        \]

    \section{Ejercicio 3}

        \textit{Calcular el coste del algoritmo ``potencia'':}

        \begin{minted}{java}
    public int potencia(int B, int N) {
        if (N == 0) return 1;
        else return B * potencia(B, N-1);
    }
        \end{minted}

        Cada recursión \texttt{N} se decrementa en 1 hasta llegar a 0, de forma 
        que es sencillo ver que para \texttt{N}$=n$ el tiempo de ejeción es:
        \[
            T(n) = n + 1 \in O(n)  
        \]

    \section{Ejercicio 4}

        \textit{Calcular el coste del algoritmo ``potencia optimizada'':}

        \begin{minted}{java}
    public int potencia2(int B, int N) {
        if (N == 0) return 1;
        int rec = potencia2(B,N/2);
        if (N%2 == 0) return rec*rec;
        else return B * rec * rec;
    }
        \end{minted}

        Este algoritmo en vez de reducir en 1 el valor de \texttt{N} se va 
        reduciendo por la mitad, de forma que el tiempo de ejecución para un 
        parametro \texttt{N} con valor $n$ será:
        \[
            T(n) = \log_2(n) + 1 \in O(\log n)
        \]

    \section{Ejercicio 5}

        \textit{Calcular el coste de invertir un número:}

        \begin{minted}{java}
    public int invertir(int n) return invertirAux(0,n);
    public int invertirAux(int ac, n) {
        if (n == 0) return ac;
        return invertirAux(ac * 10 + (n % 10), n/10);
    }
        \end{minted}

        De forma similar algunos de los algoritmos que hemos analizado hasta 
        ahora, parametro que controla el número de recursiones que hay que 
        realizar es \texttt{n} el que se divide entre 10 cada llamada 
        recursiva, de forma que el tiempo de ejecución es:
        \[
            T(n) = \log_{10} (n) + 1 \in O (\log n)
        \]
\end{document}